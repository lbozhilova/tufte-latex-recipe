\documentclass[main.tex]{subfiles}
\title{Alice's gingerbread traybake}

\begin{document}

\maketitle% this prints the handout title, author, and date

\begin{abstract}
 Mum adapted this from a Mary Berry recipe that contained treacle and a number of other ingredients we couldn`t get. She then adapted it further to fit the trays we had, which is why the quantities of ingredients are a little unusual!
\linebreak
Adapted from a Mary Berry recipe.
\end{abstract}

\section{Ingredients}

\vspace*{-\baselineskip}
\begin{table}[ht]
	\begin{tabularx}{\textwidth}{>{\hsize=0.333\hsize}X>{\bf\hsize=1\hsize}X}
	\unit[12]{oz} & golden syrup\\
	\unit[6]{oz} & muscovado sugar\\
	\unit[6]{oz} & soft margarine\\
	\unit[10]{oz} & self-raising flour \\
	\unit[1.5]{tsp} & mixed spice\\
	\unit[1.5]{tsp} & ground ginger\\
	\unit[2.5]{} & eggs \\
	\unit[3]{tbsp} & milk\\
	\end{tabularx}
\end{table}

\section{Instructions}

\begin{enumerate}
    \item Preheat the oven to \unit[160]{\textdegree C}.
    \item Grease and line a rectangular cake tin.
    \item Measure the syrup, sugar and margarine into a large pan and heat until the fat has melted.
    
    \item Remove from the heat and stir in the flour and spices.
    \item Add the eggs and milk. Beat well until smooth and pour into the prepared tin. 
    \item Bake in the pre-heated over for 40--45 minutes until the traybake is well-risen and beginning to shrink away from the sides of the tin. Allow it to cool for a few minutes before turning it out and letting it cool on a wire rack. 
	
\end{enumerate}

% \unit[325]{\textdegree F}

\end{document}