\documentclass[main.tex]{subfiles}
\title{Jo's baked goat’s cheese with balsamic onions}

\begin{document}

\maketitle

\begin{margintable}
\begin{tabularx}{\textwidth}{WZ}
Time & Makes\\ 
\midrule
30--40 minutes & 4 portions\\
\end{tabularx}
\end{margintable}

\begin{abstract}
Fancy, but simple. The perfect starter.
\end{abstract}

\section{Ingredients}

\vspace*{-\baselineskip}
\begin{table}[ht]
	\begin{tabularx}{\textwidth}{>{\hsize=0.333\hsize}X>{\bf\hsize=1\hsize}X}
	\unit[2 $\times$ 100]{g} & goat's cheese rounds\\
	\unit[50]{g} & pecans, chopped or roughly crushed\\
	\unit[25]{g} & breadcrumbs\\
	\unit[1]{} & egg, beaten\\
	\unit[3]{tbsp} & olive oil \\
	\unit[2]{} & red onions, halved and thinly sliced \\
	\unit[4]{tbsp} & balsamic vinegar \\
	\unit[4]{tbsp} & honey \\
    \unit[]{} & salt and pepper, to taste \\
    \unit[]{} & salad leaves, to serve
	\end{tabularx}
\end{table}

\section{Instructions}

\begin{enumerate}	
	\item Halve the goat's cheese horizontally.
	
	\item Mix the pecans with the breadcrumbs.
	
	\item Coat the cheese with the beaten egg and cover with the nut and crumb mixture.\marginnote{For effect, put a whole pecan on top.} 
	
	\item Bake the cheese at \unit[200]{\textdegree C} for 20 minutes.
	
	\item Meanwhile, heat the oil and add the sliced onions. Stir and cook for 10--15 minutes until soft.
	
	\item Tip in the balsamic vinegar and honey, and season. Stir over heat until syrupy.
	
	\item Serve with the cheese on top of the onions on a bed of salad. \marginnote{Slightly bitter salad leaves like rocket work best.}
	
\end{enumerate}

\end{document}

