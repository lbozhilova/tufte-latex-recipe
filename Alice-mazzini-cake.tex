\documentclass[main.tex]{subfiles}
\title{Alice's Mazzini cake (chocolate amaretti cake)}

\begin{document}

\maketitle% this prints the handout title, author, and date

\begin{margintable}
\begin{tabularx}{\textwidth}{WCZ}
Preparation & Cooking  & Serves\\ 
\midrule
30 min & 25 min & 6--8
\end{tabularx}
\end{margintable}

\begin{abstract}
We studied Italian history at A-Level, and our History teacher challenged us to make a cake counterpart to match Garibaldi biscuits. We took him up on the challenge, and found this recipe! We also came up with an entire justification as for why it could relate to Mazzini, but I cannot remember any of it! So, while its educational value is limited, I can confirm this is a good cake for occasions, including those times when you need to bribe someone who is attempting to set you homework.
\linebreak
Taken from Laura Zavan, `Italian: One Step at a Time' (2009).
\end{abstract}

\section{Ingredients}

\subsection{For the cake}
\vspace*{-\baselineskip}
\begin{table}[ht]
	\begin{tabularx}{\textwidth}{>{\hsize=0.333\hsize}X>{\bf\hsize=1\hsize}X}
	\unit[70]{g} & crunchy amaretti biscuits\\
	\unit[100]{g} & plain dark chocolate\\
	\unit[100]{g} & butter, plus extra for greasing\\
	\unit[3]{} & eggs\\
	\unit[150]{g} & caster sugar\\
	\unit[50]{g} & plain flour \\
	\unit[\nicefrac{1}{4}]{tsp} & baking powder\\
	\end{tabularx}
\end{table}

\subsection{For the ganache}
\vspace*{-\baselineskip}
\begin{table}[ht]
	\begin{tabularx}{\textwidth}{>{\hsize=0.333\hsize}X>{\bf\hsize=1\hsize}X}
	\unit[100]{g} & plain dark chocolate\\
	\unit[100]{ml} & single cream\\
	\end{tabularx}
\end{table}

\subsection{To decorate}
\vspace*{-\baselineskip}
\begin{table}[ht]
	\begin{tabularx}{\textwidth}{>{\hsize=0.333\hsize}X>{\bf\hsize=1\hsize}X}
	\unit[65]{g} & toasted almonds\\
	\end{tabularx}
\end{table}

\section{Instructions}

\begin{enumerate}
    \item Preheat the oven to \unit[180]{\textdegree C} (\unit[250]{\textdegree F} / Gas Mark 4).
    \item Generously grease a 20cm round cake tin with the extra butter. Put a pan of water on the stove and allow to simmer.
    \item Put the amaretti biscuits in a plastic fridge bag and crush to crumbs with a rolling pin.
    \item Use the amaretti crumbs to line the base of the cake tin, and chill in the fridge.
    \item Break up the chocolate and put in a bowl with the butter cut into pieces. Ensure that the water in the pan is barely simmering. If necessary, tip some out so that the water will only touch the base of the bowl when it is set above the pan. Set the bowl over the saucepan.
    \item Melt the chocolate and butter, stirring to ensure the mixture does not boil. Remove it from the heat and allow to cool. 
    \item Whisk the eggs and sugar together until they are light and creamy.
    \item Sift in the flour and the baking powder and incorporate this with a whisk. Then add the melted chocolate.
    \item  Pour the mixture into the cake tin and transfer to the oven for 25 minutes. Insert the point of a knife, or a cocktail stick, into the middle of the cake: it should come out clean. 
    \item Allow the cake to cool for at least 5 minutes before removing the sides of the tin or turning out. 
    \item To make the ganache, melt the chocolate together with the cream in a bowl set over a saucepan, as before. 
    \item Spread the ganache over the top of the cake and its sides.
    \item Decorate the cake with a circle of almonds.
	
\end{enumerate}
    
 It is possible to bake the cake the day before and make the ganache on the day it needs to be served. 

% \unit[325]{\textdegree F}

\end{document}




