\documentclass[main.tex]{subfiles}
\title{Caitlin's sour cream sugar cookies}

\begin{document}

\maketitle

\begin{abstract}
Growing up, these were a staple for any kind of holiday baking (there are cookie cutters for all seasons and we had lots of colourful sugar for decorating!). The recipe is my great aunt Karan’s recipe for cookies that are incredibly easy to make and to eat.
\end{abstract}

\section{Ingredients}

\vspace*{-\baselineskip}
\begin{table}[ht]
	\begin{tabularx}{\textwidth}{>{\hsize=0.333\hsize}X>{\bf\hsize=1\hsize}X}
	\unit[1]{cup} & sugar\\
	\unit[1]{cup} & butter\\
	\unit[1]{tsp} & vanilla extract\\
	\unit[\nicefrac{1}{4}]{tsp} & lemon extract\\
	\unit[1]{} & large egg\\
	\unit[\nicefrac{1}{4}]{cup} & sour cream\\
	\unit[3\nicefrac{1}{2}--4]{cups} & plain flour\\
	\unit[\nicefrac{1}{2}]{tsp} & salt\\
	\unit[\nicefrac{1}{2}]{tsp} & baking soda\\
	\unit[\nicefrac{1}{4}]{tsp} & nutmeg\\
	\end{tabularx}
\end{table}

\section{Instructions}

\begin{enumerate}
    \item Preheat the oven to \unit[180]{\textdegree C} / \unit[350]{\textdegree F}.
    \item Cream the sugar and the butter in a bowl until it is light and fluffy. Then add the vanilla extract and the lemon extract, before mixing in the egg and sour cream.
    \item In a bowl, combine the flour, salt, baking soda and nutmeg. Add these dry ingredients to the mixture and stir until combined.
    \item Place the dough into the fridge for an hour\marginnote{It helps to leave the dough in the fridge until you're ready to roll it out, so perhaps take \nicefrac{1}{3} of the dough out of the fridge at a time.}.
    \item Roll the dough out so that it’s about \unit[\nicefrac{1}{4}]{in} deep and cut in into shapes. Decorate with coloured sugar (if you like) before baking the cookies for 10 minutes.

\end{enumerate}

\end{document}
